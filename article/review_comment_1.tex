% Created 2018-06-25 Mo 13:07
% Intended LaTeX compiler: pdflatex
\documentclass[11pt]{scrartcl}
\usepackage[utf8]{inputenc}
\usepackage[T1]{fontenc}
\usepackage{graphicx}
\usepackage{grffile}
\usepackage{longtable}
\usepackage{wrapfig}
\usepackage{rotating}
\usepackage[normalem]{ulem}
\usepackage{amsmath}
\usepackage{textcomp}
\usepackage{amssymb}
\usepackage{capt-of}
\usepackage{hyperref}
\author{simon}
\date{\today}
\title{}
\hypersetup{
 pdfauthor={Simon Pfreundschuh},
 pdftitle={},
 pdfkeywords={},
 pdfsubject={},
 pdfcreator={Emacs 24.5.1}, 
 pdflang={English}}
\begin{document}

\setlength{\parindent}{0cm}

\section{General comments}

\subsection*{Reviewer comment 1}

1. As noted in Section 4.2.4, the a priori assumptions do not describe reality
very well. In particular, I suspect that the information content of Dm and N0* is
highly dependent on the a priori assumptions of these two variables in the
retrieval framework. Especially with a radar measurement, since Z is sensitive to
both parameters over a wide range of the parameter space, the relative
sensitivity and therefore information content will almost entirely depend on the
relative constraints on these parameters imposed by Xa and Sa. As such it is
imperative to accurately characterize these. I understand the choice to use the
DARDAR constraints, but it’s clear from the cross-section plots that the model
ice particle concentrations vary over a much wider range than the roughly 2
orders of magnitude that Eq. 4 provides over a 220-272 K temperature range.
So, when the retrieval results are compared to model “reality”, it seems that a
lot of N0* variability is folded into Dm and this is especially evident in
Figures 13 and 14. My overall concern is that it is difficult to interpret some
of the results when the model fields and the a priori assumptions differ so
strongly.

\subsubsection*{Author response:}

It should be clarified here that Eq. (4) only gives the variation of the mean
profile of the a priori for $N_0^*$ and that the standard deviation for $N_0^*$
has been set to a value of $2$ in log-space, allowing $N_0^*$ to vary over several
orders of magnitude.

Nonetheless, the point raised by the referee remains valid. Since the topic
of the study are synergies between radar and radiometer observations, we aimed
to keep the a priori assumptions realistic (the DARDAR mean profile) but at the
same time sufficiently loose (high std. dev.) in order to not introduce information
a priori that may be provided by the synergy. To clarify this, the following
lines haves been added in the introduction:

\vspace{1em}

\textit{The standard deviations in the covariance matrix have been deliberately
  chosen to be very loose in order avoid the introduction of constraining
  information that may be provided by the radiometer observations.}

\subsection*{Reviewer comment 2}

Forward model error is introduced when the different species present in the
model microphysics are combined into one species and when different scattering
models are used to represent the ice particles. That this is not represented in
Se could lead to over-fitting and poor convergence (I suspect this is part of
the reason why the normalized cost is much higher for the radiometer-including
retrievals). It should be relatively easy to quantify this error by re-running
the simulations with the retrieval assumptions(combining ice species, different
scattering models), and I suspect that this error term would dominate the
instrument noise term for many channels.

\subsubsection*{Author response}

This is certainly another valid point and has been addressed in the way suggested
by the authors. All calculations were repeated and the corresponding results
updated in the manuscript. To clarify this a sub-section specifying the assumed observation
errors has been added to Sect. 2.3..
\vspace{1em}


\section{Specific comments}

\subsection*{Reviewer comment 1}
Lines  85-88:   I  recommend  the  use  of  geographical  spatial  references  (i.e.,north/south rather than left/right)

\subsubsection*{Author response}

The proposed change is adopted in the revised version of the manuscript and the corresponding
sentences now read as follows:

\textit{
The first test scene, shown in panel (a), is located in the
tropical Pacific and contains a convective storm system in the northern half of
the scene and its anvil that extends into the southern half of the scene. The
second scene, shown in panel (b), is located in the North Atlantic and contains
an ice cloud in the southern part and a low-level, mixed-phase cloud in the
remainder of the scene.
}



\subsection*{Reviewer comment 2}
Line  98  (also  176,252,449):  Instead  of  vertical/horizontal  (which  are  dependentone the convention used for plotting), I recommend the use of concentration/size tocharacterize the dimensions of the particle size distribution.

\subsubsection*{Author response}

The proposed changes will be adopted in the revised version of the manuscript
by introducing the following changes:

L. 98:

\textit{
As these plots show, the
assumed particle size distributions across different ice species vary mostly in
their scaling with respect to size and concentration, whereas the function shape shows less
variability.}

L. 176:

\textit{ The PSD of a hydrometeor species at a given height level is represented
  by two scaling parameters: The mass-weighted mean diameter $D_m$, which scales
  the size of the particles, and the normalized number density $N_0^*$, which
  scales the concentration of particles.}

L. 252:

\textit{ In the figure, the cloud signal is displayed in $D_m$-mass density
  space and thus shows how the measured passive cloud signal varies with size
  and concentration of particles in the cloud. }


L. 449:
\textit{The results show that the combined observations can simultaneously
  constrain the size and concentration of particles in the cloud.}


In addition to this, also the following sentences which referred to the horizontal and vertical
scaling of the PSD will be changed accordingly:

L. 183:

\textit{ The retrieved scaling parameters of particle size and concentration,
  $D_m$ and $N_0^*$, are used as units for the axes of the plot so that the
  shape of the PSD becomes independent of the retrieved mass density and number
  concentration. }

L. 244:

\texit{The question that is addressed here is whether the combination of active
  and passive observations is able to constrain both the size and concentration
  of the ice particles in the cloud.}

\subsection*{Reviewer comment 3}

Line 100: A few more details on the Milbrant and Yau microphysics sheme that are relevant to this study would be helpful here. For example: What is the assumed shape(functional form) of the particle size distribution, and what are the prognostic variables(e.g., number concentration, mixing ratio)?

\subsubsection*{Author response}

To address the reviewers comment, the paragraph describing the Milbrandt-Yau microphysics scheme
has been rewritten and now reads as follows:

\textit{
The GEM model uses a two-moment scheme with six types of hydrometeors to
represent clouds and precipitation \citep{milbrandtyau05}: Two classes of liquid
hydrometeors (rain and liquid cloud) and four of frozen hydrometeors (cloud ice,
snow, hail and graupel). The particle size distribution (PSD) of each
hydrometeor class is described by a three-parameter gamma distribution. The
prognostic parameters of the two-moment scheme are the slope and intercept
parameters of the distribution, which are derived from the mixing ratios and
number densities predicted by the GEM model. The third parameter of the PSD and
the mass-size relationship of each hydrometeor class are set to fixed,
class-specific values. The parameters of the mass-size relationships are given
in Tab.~\ref{tab:species_parameters}. The masses of all ice particles in the
model are assumed to scale with a power of three, which leads to high densities
for large particles.
  }

\subsection*{Reviewer comment 4}
Line 135: Does the ARTS radar solver also provide analytic Jacobians?

\subsubsection*{Author response}
Yes, it does. This is now also mentioned in the description of the retrieval forward model:

\textit{
All simulations presented in this study were performed using Version 2.3.1245 of
the Atmospheric Radiative Transfer Simulator (ARTS, \cite{arts18}). Radar
reflectivities are computed using ARTS' built-in single-scattering radar solver,
which provides analytic Jacobians. For the ...
}

\subsection*{Reviewer comment 5}
Line 187: “particles” should be “particle”

\subsubsection*{Author response}

This will be corrected in the revised version of the manuscript.


\subsection{Reviewer comment 6}
Line 198: Is Dm also only retrieved at these 10 points, or just N0* (and Dm retrievedin each radar range gate as in Grecu et al. 2016)?

\subsubsection*{Author response}

$D_m$ is actually retrieved at the resolution of the GEM model scenes. To make this more clear
the following sketch has been created and will be included in the beginning of the section
describing the retrieval setup:

\begin{figure}
\centering
\includegraphics[width = 0.8\linewidth]{../plots/retrieval_sketch}
\caption{Illustration of the retrieval quantities and their respective retrieval
  grids. Grey, dashed lines show the resolution of the GEM model data while the
  filled circles represent the grid points of the different retrieval
  quantities. Panel (a) shows the configuration used in the radar-only and
  combined retrieval. Panel (b) shows the configuration used in the passive-only
  retrieval}
\label{fig:retrieval_sketch}
\end{figure}



\subsection{Reviewer comment}
7.  Line 256:  Actually, this is only one example of how the radar and radiometer measurements can be complementary.  Even if the lines were parallel (and thus no infor-mation distinguishing size from concentration could be obtained), the radar still locates the cloud and describes its vertical structure.  One can imagine a cloud of the same ice water path and particle size at two different heights having different brightness temperatures due to changes in the water vapor absorption above the cloud – having theradar information would provide increased information content about the ice water pathin this case than the radiometer measurement alone.

\subsubsection*{Author response}
Although this is certainly one advantage of combining radar and radiometer
observations, the authors would not consider this a true synergy of the active
and passive observations since correctly locating the cloud in the atmosphere is
something that the radar can do alone. For the passive to add benefit to the
radar , the combined observations must provide additional information on the
microphysics that the radar alone does not provide. To better clarify this
reasoning the following sentence will be added to the manuscript:

\subsection*{Reviewer comment 8}
Table 4: Why are the values for GemSnow and GemGraupel different than in Table1?

\subsubsection*{Author response}

The differences in the reported values for GemSnow and GemGraupel were due to a mistake
made by the authors. The values will be corrected in the updated version of the manuscript.

\subsection*{Reviewer comment 9}
Figures 7 and 8:  I’m not sure why these are separate figures – it seems like allpanels could fit on one page.

\subsubsection*{Author response}
The figures will be combined into a single figure.

\subsection*{Reviewer comment 11.}
 Line 374: recommend using “represent” instead of “predict”

\subsubsection*{Author response}
The proposed change will be adopted in the updated version of the manuscript.

\subsection*{Reviewer comment 12}
 Line 382: should be “reference” instead of “references”

\subsubsection*{Author response}
This will be corrected in the updated version of the manuscript.

\subsection*{Reviewer comment 13}

Line  414:  How  are  the  truncated  PSDs  (using  GemSnow)  represented  in  theforward simulations? Is total ice water content conserved? If so, how is it spread amongthe valid particle sizes – equally, or is the truncated mass allocated to the smallest size bin?

\subsubsection*{Author response}
Total IWC is not conserved in the handling of PSDs. The point raised by the reviewer has been
investigated by assessing the effect of the truncation of the IWC. The results of the analysis
are given in the figure below. As the results show, the truncation can introduce significant
errors. This, however, is only the case when the GemSnow particle is used to represent all
ice species. If it is used only to represent snow, the errors are negligible (not shown). Because
of that as well as another comment regarding the choice of the tested particles, the selection of
tested particles will be modified and the GemSnow particle will not be included.

\begin{figure}[!hbpt]
  \centering
  \includegraphics[width = 1.0\textwidth]{../plots/truncated_iwc}
  \caption{Joint distribution of truncated and true ice water content (IWC) for the
    two test scenes.}
\end{figure}

\end{document}
