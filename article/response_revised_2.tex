
\section{General comments}

\subsection*{Reviewer comment 1}

L 106: that cloud ice particles are small and abundant while snow particles are
large and much rare is nature not the model. So I would recommend to change the
sentence: " An important characteristic can be identified here.."

\subsubsection{Author response}

The proposed change will be adopted in the revised manuscript.

\subsubsection*{Changes in manuscript}

  \begin{change}[106]
    An important characteristic \DIFdelbegin \DIFdel{of the model }\DIFdelend
    can be identified here, which will help to better understand the retrieval
    results presented later:
  \end{change}

\subsection*{Reviewer comment 2}

 L222-223: The two sentences are a bit contradictiry: Just say M1 results will be shown for certain aspects only

\subsubsection{Author response}

The proposed changes will be adopted in the revised manuscript.

\subsubsection*{Changes in manuscript}

  \begin{change}[220]
 In addition to a two-moment radar-only retrieval, also a one-moment version
 (M1), in which only the $D_m$ parameter is retrieved has been tested.
 \DIFdelbegin \DIFdel{For completeness, retrieval results for IWC will be
   reported also for the M1 version. However, to allow for better comparison
   with the combined and passive-only retrieval , for the remaining results only
   the two-moment version is considered}\DIFdelend \DIFaddbegin \DIFadd{However,
   results of this version will be shown only for the comparison of IWC
   retrieval errors}\DIFaddend .
  \end{change}

\subsection*{Reviewer comment 3}

 L247 WC is used as abbreviation before being defined

\subsubsection{Author response}

We will correct this in the revised manuscript.

\subsubsection*{Changes in manuscript}

  \begin{change}[247]
 Mass backscattering efficiency and attenuation coefficient are defined as the
 ratio of the corresponding cross-section $\sigma$ and the bulk water content
 \DIFaddbegin \DIFadd{(\text{WC})}\DIFaddend :
\begin{align}
  Q &= \frac{\sigma}{\text{WC}}
\end{align}
\end{change}

\subsection*{Reviewer comment 4}

 Figure 5 caption: say that this is an ice cloud

\subsubsection{Author response}

The caption of Fig.~5 will be changed in the revised manuscript as shown below.

\subsubsection*{Changes in manuscript}
 

\begin{change}[284]
Simulated observations of a homogeneous, $5\ \unit{km}$ thick \DIFaddbeginFL \DIFaddFL{ice }\DIFaddendFL cloud
  \DIFdelbeginFL \DIFdelFL{layer }\DIFdelendFL centered at $10\ \unit{km}$ with varying water content $m$ and
  mass-weighted mean diameter $D_m$. The panels display the maximum radar
  reflectivity in dBZ ($\text{dBZ}_\text{max}$) overlaid onto the cloud signal
  ($\Delta T_B$) measured by selected radiometer channels of the MWI (first row)
  and ICI radiometers (second row).
  \end{change}

\subsection*{Reviewer comment 5}

I really like Fig. 12 It shows the clear contribution in DOF from the different
parameters. Just as a quick idea which does not neccessaryl need to be
implemented but might strengthen the discussion on LCWC: Could you look at the
ratio of the the combined DOF and the sum of the the single retrievals. This
could help to explain that the ice information is in both radar and passiv and
therefore in the combined retrieval the nicrowave information content for ice is
not needed (ice contribution is determined) and therefore the information
content is transfered to LCWC is transfer. This argumention in 4.2.3 is
currenttly not too strong.

\subsubsection{Author response}

As suggested by the referee, we have produced a plot of the ratio of the DFS of
the combined retrieval and the sum of DFS of the single-instrument retrievals.
Since this plot indeed strengthens our arguments on the combined information
content on LCWC, we have extended the discussion in Sect. 4.2.3 as shown below.

\begin{figure}
  \centering
  \includegraphics[width=\textwidth]{../plots/dfs_ratios}
  \caption{Ratios of the DFS of the combined retrieval ($\text{DFS}_\text{co}$) and
    the sum of the DFS or the single-instrument retrievals ($\text{DFS}_\text{ro} + \text{DFS}_\text{po}$)
    for the two test scenes.}
  \label{fig:dfs_ratios}
\end{figure}
j
\subsubsection*{Changes in manuscript}

\begin{change}[523]
 \DIFaddbegin \DIFadd{This effect is more clearly visible when the fraction of total
DFS of the combined retrieval and the sum of the DFS of the single-instrument retrieval is considered, as shown in Fig.~\ref{fig:dfs_ratios}. This ratio,
which gives an estimate of the synergistic information content in the active
and passive observation, shows a distinct increase in the second test scene
at around $42^\circ\ N$, where the scene contains a mixed-phase cloud.
}
 \end{change}


\subsection*{Reviewer comment 6}

 L548: "..was able TO reproduce.. 


\subsubsection*{Author response}
The missing word will be added to the revised manuscript.

\subsubsection*{Changes in manuscript}

\begin{change}[548]
Moreover, the combined retrieval showed clear sensitivity to particle number
concentrations and was able \DIFaddbegin \DIFadd{to }\DIFaddend reproduce their
vertical structure in regions where the cloud composition ...
\end{change}

\subsection*{Reviewer comment 6}

 Table 3 caption - give symbol lq for correlation length

\subsubsection*{Author response}

The caption of Table~3 will be corrected in the revised manuscript, as shown
below.

\subsubsection*{Changes in manuscript}

\begin{change}[218]
A priori uncertainties \DIFaddbeginFL \DIFaddFL{$sigma_q$ }\DIFaddendFL and correlation lengths \DIFaddbeginFL \DIFaddFL{$l_q$ }\DIFaddendFL used in the retrieval.
\end{change}
